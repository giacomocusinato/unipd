\section{Descrizione del sistema floating-point}

Il sistema floating-point è 	un metodo di rappresentazione numerica utilizzato dai calcolatori per rappresentare numeri razionali ed approssimazioni dei numeri reali in modo tale da ottenere un compromesso tra range e precizione. La rappresentazione in floating-point normalizzata in base 2 è quella utilizzata dallo standard IEEE: i numeri sono scritti nrlla forma \(x={f2}^e \) dove \(f=\pm 1.f_{-1}f_{-2}...f{-n}  \) e \(e=\pm e_{Ne-1}e_{Ne-2}...e{0} \) sono le cifre rispettivamente della mantissa e dell'esponente, in numero finito e valiri binari. La rappresentazione floating point può essere espressa così:

\addvbuffer[12pt 12pt]{\begin{tabular}{|l|l|l|l|l|l|l|l|l|l|l|l|l|l|}
\hline
s & e & e & e & e & .... & m & m & m & m & m & m & m & .... \\
\hline
\multicolumn{1}{c}{\upbracefill} & \multicolumn{5}{c}{\upbracefill}& \multicolumn{8}{c}{\upbracefill}\\[-1ex]
\multicolumn{1}{c}{$\scriptstyle segno$} & \multicolumn{5}{c}{$\scriptstyle esponente$}& \multicolumn{8}{c}{$\scriptstyle mantissa$}\\
\end{tabular}}

Nel sistema IEE, la rappresentazione in singola precisione è a 32bit con (1 bit per il segno, 8 per l'esponente e 23 bit per la mantissa) mentre quella in doppia precisione viene rappresentata in 32 bit (1 bit per il segno, 11 bit per l'esponente e 52 bit per la mantissa).

